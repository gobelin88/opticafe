\documentclass{article}

\usepackage{color}
\usepackage[usenames,dvipsnames]{xcolor}
\usepackage[latin1]{inputenc}
\usepackage{graphicx}
\usepackage{geometry}
\usepackage{listings}
\usepackage{hyperref}
\usepackage{amsmath,amsfonts,amssymb}
\geometry{hmargin=1.0cm,vmargin=1.5cm}

\lstset{language=C++,
                basicstyle=\ttfamily,
                keywordstyle=\color{blue}\ttfamily,
                stringstyle=\color{red}\ttfamily,
                commentstyle=\color{magenta}\ttfamily,
                morecomment=[l][\color{magenta}]{\#}
}

\hypersetup{
    colorlinks=true,
    linkcolor=black,
}

\begin{document}

\title{Opticafe - Ref�rence Guide}
\author{Lo�c Huguel}

\maketitle
\tableofcontents

\newpage
\section{Deeling with Opticafe}

\subsection{Introduction}

Opticafe is design to solve Non-Linear probl�mes in order to fit a \textbf{model to data} or \textbf{find non trivial zeros} of a funtion. It implements severals famous algorithms (Gauss-Newton,Levenberg-Marquards or DogLeg) to optimise a system of equations witch the global form are:

\subsubsection{With external measures}
\[
	\hat{Y_i}=f(P,\hat{X_i}) 
\] 

Where
\begin{itemize}
\item $\hat{Y_i}$ are the outputs vectors of measures provide by user (see \ref{step2})
\item $\hat{X_i}$ are the inputs vectors of measures provide by user (see \ref{step2})
\item $P$ the vector of parameters you're trying to optimise.
\end{itemize}

\begin{center}
\fbox{In this case Opticafe will try to minimise : $\sum |f(P,\hat{X_i})-\hat{Y_i}|^2 = 0$}
\end{center}

\subsubsection{Without external measures}
\[
	Y=f(P) 
\] 

Where
\begin{itemize}
\item $Y$ is the output vector
\item $P$ the vector of parameters you're trying to optimise.
\end{itemize}

\begin{center}
\fbox{In this case Opticafe will guess the zeros of the function ie it will try to minimise : $|f(P)|^2=0$}
\end{center}

\subsection{Step 1: Define a system of equations}

So the first step consist in writing your system of equations. Write "yj" to specifie the component "j" of the Y vector($j\in[0,9]$). Write "xk" to specifie the component "k" of the X vector ($k\in[0,9]$). Write "pl" to specifie the component "l" of the P vector ($k\in[0,9]$). \\	

For instances:\\
\begin{center}
\begin{tabular}{|ll|}
\hline
y0=sin(p0) & $R^1\Rightarrow R^1 $ with no measures\\
\hline
y0=sin(p0) & $R^1\Rightarrow R^2 $ with no measures\\
y1=cos(p0) &\\
\hline
y0=sin(p0+p1) & $R^2\Rightarrow R^1 $ with no measures\\
\hline
y0=sin(p0+p1*x0) & $R^2\Rightarrow R^1 $ with measures\\
\hline
y0=sin(p0+p1*x0) & $R^2\Rightarrow R^2 $ with measures\\
y1=cos(p0+p1*x0) &  \\
\hline
y0=sin(p0+p1*x0) & $R^3\Rightarrow R^3 $ with measures\\
y1=cos(p0+p1*x0)-1 &  \\
y2=tan(p0+p1/p3*x0+x2)& \\
\hline
\end{tabular}
\end{center}

\newpage
\subsection{Step 2: Loading measures}
\label{step2}
Maybe you have define a system with measures so it may be useful to load them. To deel with this you just have to write $data=Path/measures.txt$ after you defined the system.\\

The format of measures.txt as to be:

\begin{center}
\begin{tabular}{|l|}
\hline
y1;y2;...;yn ; x1;x2;...;xn\\
y1;y2;...;yn ; x1;x2;...;xn\\
y1;y2;...;yn ; x1;x2;...;xn\\
y1;y2;...;yn ; x1;x2;...;xn\\
....\\
\hline
\end{tabular}
\end{center}

And must match the system defined, so for instance if you have defined:

\begin{center}
\begin{tabular}{|l|}
\hline
y0=sin(p0+p1*x0)\\
y1=cos(p0+p1*x0)-1 \\
y2=tan(p0+p1/p3*x0+x2)\\
data=./measures.txt\\
\hline
\end{tabular}
\end{center}

You need to measure file witch could be:

\begin{center}
\begin{tabular}{|l|}
\hline
0;0;0 ; 1;2;2\\
1;0;0 ; 1;2;2\\
1;1;0 ; 1;2;2\\
0;0;0 ; 1;2;3\\
\hline
\end{tabular}
\end{center}


\subsection{Step 3: Minimise the system of equations}

Now you want to minimise or find the roots. The most important step is to define the initial vector of parameters (p\_init). Just write "p\_init=0,0" if $P$ has to dimensions. If non specified p\_init=0.\\

\noindent
S�lect an algorithm in the menu, press "Minimise" and see the results.\\

For instance:

\section{Simples Exemples}
\subsection{Linear regression}
\subsection{Polynomial regression}
\subsection{Multiple regression}

\section{Advanced Exemples}
\subsection{Rosenbrock function}
\subsection{Newton optimisation}


\end{document}